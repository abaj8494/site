\begin{center}\underline{\uppercase{fouls}}\end{center}
    \begin{itemize}[noitemsep]
    \tiny

    \item 15.1.1. A player intentionally initiating minor contact is still a breach of the rules, but is to be treated as a violation, and not a foul.
    \item 15.4. Only the player fouled may claim a foul, by calling “Foul”
    \item 15.5.1. any offensive player may call a double team, and any defensive player may call a travel infraction.
    \item 17.1.1 \textbf{Dangerous Play:} Actions demonstrating reckless disregard for the safety of fellow players, or posing significant risk of injury to fellow players, or other dangerously aggressive behaviours,
    \item 17.2.1. A \textbf{Receiving Foul} occurs when a player initiates non-minor contact with an opponent before, while, or directly after, either player makes a play on the disc.
    \item 17.3.1. A \textbf{Strip Foul} occurs when an opponent fouls a player and that causes the player to drop a disc they caught or to lose possession of the disc.
        \begin{itemize}
            \item 17.3.2. If the reception would have otherwise been a goal, and the foul is accepted, a goal is awarded.
        \end{itemize}
    \item 17.4.1. A \textbf{Blocking Foul} occurs when a player takes a position that an opponent moving in a legal manner will be unable to avoid, taking into account the opponents expected position based on their established speed and direction, and non-minor contact results. This is to be treated as either a receiving foul or an indirect foul, whichever is applicable.
    \item 17.5.1. A \textbf{Force-out Foul} occurs when a receiver is in the process of establishing possession of the disc, and is fouled by a defensive player before establishing possession, and the contact caused the receiver:
        \begin{itemize}
            \item 17.5.1.1. to become out-of-bounds instead of in-bounds; or
            \item 17.5.1.2. to catch the disc in the central zone instead of their attacking end zone.
            \item 17.5.2. If the receiver would have caught the disc in their attacking end zone, it is a goal;
            \item 17.5.3. If the force-out foul is contested, the disc is returned to the thrower if the receiver became out-of-bounds, otherwise the disc stays with the receiver.
        \end{itemize}
    \item 17.6.1. A \textbf{Defensive Throwing Foul} occurs when:
        \begin{itemize}
            \item 17.6.1.1. A defensive player is illegally positioned (Section 18.1), and there is non-minor contact between the illegally positioned defensive player and the thrower; or
            \item 17.6.1.2. A defensive player initiates non-minor contact with the thrower, or there is non-minor contact resulting from the thrower and the defender both vying for the same unoccupied position, prior to the release.
            \item 17.6.1.3. If a Defensive Throwing Foul occurs prior to the thrower releasing the disc and not during the throwing motion, the thrower may choose to call a contact infraction, by calling “Contact”. After a contact infraction that is not contested, play does not stop and the marker must resume the stall count at one (1).
        \end{itemize}
    \item 17.7.1. An \textbf{Offensive Throwing Foul} occurs when the thrower is solely responsible for initiating nonminor contact with a defensive player who is in a legal position.
        \begin{itemize}
            \item 17.7.2. Contact occurring during the thrower's follow through is not a sufficient basis for a foul, but should be avoided.
        \end{itemize}
    \item 17.8.1. An \textbf{Indirect Foul} occurs when there is non-minor contact between a receiver and a defensive player that does not directly affect an attempt to make a play on the disc.
        \begin{itemize}
            \item 17.8.2. If the foul is accepted the fouled player may make up any positional disadvantage caused by the foul.
        \end{itemize}
    \item 17.9. \textbf{Offsetting Fouls}:
        \begin{itemize}
            \item 17.9.1. If accepted fouls are called by offensive and defensive players on the same play, these are offsetting fouls, and the disc must be returned to the last non-disputed thrower.
            \item 17.9.2. If there is non-minor contact that is caused by two or more opposing players moving towards a single point simultaneously, this must be treated as offsetting fouls.
            \item 17.9.2.1. However if this occurs after the disc has been caught, or after the relevant player/s involved can no longer make a play on the disc, this must be treated as an Indirect Foul (excluding contact related to Section 17.1).
        \end{itemize}
\end{itemize}
\begin{center}[4]\end{center}
