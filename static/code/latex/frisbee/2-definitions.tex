\begin{flushright}
\begin{minipage}{9cm}
    \begin{flushleft} \emph{If you wish to converse with me define your terms.} \end{flushleft}
    \begin{flushright}--- Voltaire\end{flushright}
\end{minipage}
\end{flushright}

\small
\begin{enumerate}
    \item \textbf{call}: a clearly communicated statement that a foul, infraction, violation, turnover or injury has occurred.
    \item \textbf{catch}: a non-spinning disc trapped between at least two body parts.
    \item \textbf{central zone (CZ)}: the area of the playing field including the goals lines but excluding the end zones and the perimeter lines.
    \item \textbf{end zone (EZ)}: the section of the field within which catching the disc results in a goal.
    \item \textbf{minor-contact}: contact that does not alter the movements or position of another player.
    \item \textbf{stoppage}: any halting of play due to a foul, violation, discussion, contested call, injury or time-out, that requires a check to restart play.
    \item \textbf{foul}: a breach of the rules due to non-minor contact between two or more opposing players (15.1).
    \item \textbf{violation}: every breach of the rules excluding fouls and infractions is a violation (15.3).
    \item \textbf{infraction}: a breach of the rules regarding marking or travels.
    \item \textbf{straddle}: usually refers to a non-goal where the offensive player has one foot on either side of the goal line. also see the straddle marking infraction (18.1.1.2).
\end{enumerate}

\begin{center}[2]\end{center}
