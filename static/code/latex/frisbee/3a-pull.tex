\underline{\uppercase{the pull}}
\begin{itemize}[noitemsep]
\tiny
    \item [7.3] After signalling readiness all offensive players must stand with one foot on their defending goal line without changing location relative to one another until the pull is released.
    \item [7.4] After signalling readiness all defensive players must keep their feet entirely behind the vertical plane of the goal line until the pull is released.
    \item [7.5] If a team breaches 7.3 or 7.4 the opposing team may call a violation (“offside”). This must be called before the offence touches the disc (7.8 still applies).
        \begin{itemize}
            \item[7.5.1] If defense calls offside then play continues as if a time-out had been called. 
            \item[7.5.2] If offence calls offside then play starts from the brick mark.
        \end{itemize}
    \item[7.8] It is a turnover if offence touches the pull without establishing possession.
    \item[7.9] If an offensive player catches the pull they must establish a pivot point at that location, \textit{even if that pivot point is in their defending end zone.}
    \item[7.11] If the disc initially contacts the playing field and then becomes out-of-bounds without contacting an offensive player, the thrower must establish a pivot point where the disc first crossed the perimeter line, or the nearest location in the central zone if that pivot point would be in their defending end zone
    \item[8.4] Any player may attempt to stop a disc from rolling or sliding after it has hit the ground.
\end{itemize}
\vspace*{-0.5cm}
\begin{center}[a]\end{center}
