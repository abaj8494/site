You now know addition, subtraction and multiplication. It is time to realise that $7\times 5 - 4$ can be interpretted in \emph{two} different ways. Only one of these ways is correct.

$7\times 5-4$ is:
\begin{solutionordottedlines}[1in]
    Equal to 31. It is not equal to 7 because we must do the multiplication first.
\end{solutionordottedlines}

\bigskip
\fbox{
    \parbox{\linewidth}{
        {\Large \textbf{BODMAS:} =

    \textbf{B}\fillin[rackets] \textbf{O}\fillin[pen] \textbf{D}\fillin[ivision] \\\textbf{M}\fillin[ultiplication] \textbf{A}\fillin[ddition] \textbf{S}\fillin[ubtraction]}
}}
\bigskip
\begin{instructor}
    BODMAS is an acronym for: \textbf{B}rackets \textbf{O}pen \textbf{D}ivision \textbf{M}ultiplication \textbf{A}ddition \textbf{S}ubtraction
\end{instructor}

\begin{examples}
    \begin{questions}
        \Question[1] $5 \times 3+2 =\fillin[17]$
        \Question[1] $5\times (3+2) = \fillin[25]$
        \Question Carry out each of the following calculations.
        \begin{parts}\begin{multicols}{2}
            \part \(3 \times 7-4=\fillin[17]\)
            \part \(3 \times(7-4)=\fillin[9]\)
            \part \(5 \times 6+8=\fillin[38]\)
            \part \(7 \times(11+4)=\fillin[105]\)
            \part \(3+4 \times 2=\fillin[11]\)
            \part \(25-6 \times 3=\fillin[7]\)
        \end{multicols}\end{parts}
    \end{questions}
\end{examples}\marginpar{\scriptsize{}notice that we used brackets to force the addition to occur first.}

\subsection{The Distributive Law}
This is another concept as important as \textbf{BODMAS}:

When multiplying, it is sometimes useful to express one of the numbers you are multiplying as a sum of two other numbers. For example:

\[
\begin{aligned}
6 \times 105 & =6 \times(100+5) \\
& =6 \times 100+6 \times 5 \\
& =600+30 \\
& =630
\end{aligned}
\]

This is an example of the distributive law for multiplication over addition. Using the distributive law can often help you to do multiplications more easily.

Using the distributive law for multiplication over subtraction can also help to make a multiplication easier. For example:

\[
\begin{aligned}
85 \times 98 & =85 \times(100-2) \\
& =(85 \times 100)-(85 \times 2) \\
& =8500-170 \\
& =8330
\end{aligned}
\]

\begin{examples}
    \begin{questions}
        \Question[3] Carry out each of the following computations, using the distributive law.
        \begin{parts}
            \part \(106 \times 8=\fillin[848]\)
            \part \(43 \times 7+43 \times 3=\fillin[430]\)
            \part \(97 \times 88=\fillin[8536]\)
        \end{parts}
        \Question[4] For each of the following, put a whole number in the box to make the statement true.
        \begin{parts}
            \part \(6 \times(7+\square)=6 \times 7+6 \times 5\)\\
            \part \(13 \times 7+13 \times 8=\square \times(7+8)\)\\
            \part \(10 \times(4+7)=10 \times 4+10 \times \square\)\\
            \part \(8 \times \square=8 \times 10+8 \times 7\)
        \end{parts}
    \end{questions}
\end{examples}

\begin{exercises}
    \begin{questions}
        \Question[9] Carry out these calculations mentally.
        \begin{multicols}{2}
        \begin{parts}
            \part \(5 \times 6-3=\fillin[]\)
            \part \(5 \times(6-3)=\fillin[]\)
            \part \(6 \times 5+7=\fillin[]\)
            \part \(6 \times(5+7)=\fillin[]\)
            \part \(11 \times(2+3)=\fillin[]\)
            \part \(10+2 \times 7=\fillin[]\)
            \part \((10+2) \times 7=\fillin[]\)
            \part \(19-9 \times 2=\fillin[]\)
            \part \((19-9) \times 2=\fillin[]\)
        \end{parts}
        \end{multicols}
        \Question[6] Carry out these calculations mentally, using the distributive laws.
        \begin{parts}
            \part \(6 \times 87+4 \times 87=\fillin[]\)
            \part \(64 \times 77+36 \times 77=\fillin[]\)
            \part \(23 \times 78+77 \times 78=\fillin[]\)
            \part \(27 \times 4=\fillin[]\)
            \part \(9 \times 102=\fillin[]\)
            \part \(87 \times 101=\fillin[]\)
        \end{parts}
        \Question[4] Put a whole number in the box to make each statement true.
        \begin{parts}
            \part \(11 \times(7+\square)=11 \times 7+11 \times 5\)
            \part \(15 \times 7+15 \times 8=\square \times(7+8)\)
            \part \(11 \times \square=11 \times 20+11 \times 3\)
            \part \(21 \times 6+21 \times 8=\square \times(6+8)\)
        \end{parts}
        \Question[4] Use the distributive law to carry out these calculations.
        \begin{parts}
            \part \(6 \times 87-4 \times 87\)
            \begin{solutionordottedlines}[1in]
            \end{solutionordottedlines}
            \part \(123 \times 77-23 \times 77\)
            \begin{solutionordottedlines}[1in]
            \end{solutionordottedlines}
            \part \(23 \times 78-13 \times 78\)
            \begin{solutionordottedlines}[1in]
            \end{solutionordottedlines}
            \part \(8 \times 120-13 \times 120\)
            \begin{solutionordottedlines}[1in]
            \end{solutionordottedlines}
        \end{parts}
        \Question[4] Put a whole number in each box to make the statements true.\\
        \begin{parts}
            \part \(11 \times(7-\square)=11 \times 7-11 \times 5\)\\
            \part \(15 \times 8-15 \times 7=\square \times(8-7)\)\\
            \part \(9 \times \square=9 \times 20-9 \times 1\)\\
            \part \(8 \times \square=8 \times 100-8 \times 1\)
        \end{parts}
        \Question[2] There are 40 passengers on a bus when the bus stops. Eleven passengers leave the bus and 7 passengers get on the bus. How many passengers are there on the bus?
            \begin{solutionordottedlines}[2in]
            \end{solutionordottedlines}
        \Question[2] Daniel has 6 boxes of chocolates, each containing 20 chocolates, and he also has 54 loose chocolates. How many chocolates does he have altogether?
            \begin{solutionordottedlines}[2in]
            \end{solutionordottedlines}
        \Question[] Deborah has 5 hair clips while Christine has 11.
        \begin{parts}
            \Part[1] What is the total number of hair clips?
            \begin{solutionordottedlines}[1in]
            \end{solutionordottedlines}
            \Part[2] Jane has three times as many hair clips as Deborah, and Leanne has three times as many hair clips as Christine. What is the total number of hair clips that Jane and Leanne have?
            \begin{solutionordottedlines}[1in]
            \end{solutionordottedlines}
        \end{parts}
        \Question[] John and Minh walk each day for 12 days to keep fit. John walks \(9 \mathrm{~km}\) a day, and Minh walks \(5 \mathrm{~km}\) a day.
        \begin{parts}
            \Part[2] What is the total distance walked by John and Minh in the 12 days?
            \begin{solutionordottedlines}[1in]
            \end{solutionordottedlines}
            \Part[2] How many more kilometres than Minh has John walked in the 12 days?
            \begin{solutionordottedlines}[1in]
            \end{solutionordottedlines}
        \end{parts}
        \Question[3] James has 10 fewer novels than Janine, and Ainesh has 5 times the number of novels that James has. Janine has 13 novels. How many novels does Ainesh have?
            \begin{solutionordottedlines}[2in]
            \end{solutionordottedlines}
        \Question[3] Becky earns \(\$ 5\) less than Ben each week, and Jake earns 4 times the amount that Becky earns each week. Ben earns \(\$ 17\) each week. How much does Jake earn each week?
            \begin{solutionordottedlines}[2in]
            \end{solutionordottedlines}
    \end{questions}
\end{exercises}
