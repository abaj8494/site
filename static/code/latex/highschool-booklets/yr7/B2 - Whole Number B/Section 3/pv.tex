The symbols \(0,1,2,3,4,5,6,7,8\) and 9 are called digits. For example, 421 is a three-digit number and 40000 is a five-digit number.

The place value of a digit in a number means its value according to its place in that number.

We can break apart any number and write it showing its place-value parts. For example:

\[
\begin{aligned}
3721 & =3 \times 1000+7 \times 100+2 \times 10+1 \\
& =3000+700+20+1
\end{aligned}
\]

For the number 3721, we say that:

\begin{itemize}
  \item the place value of 3 is 3000
  \item the place value of 7 is 700
  \item the place value of 2 is 20
  \item the place value of 1 is 1 .
\end{itemize}

A number can be represented in a place-value table. In this table, the places are thousands, hundreds, tens and ones. The ones place is sometimes called the units place.

\begin{center}
\begin{tabular}{|c|c|c|c|}
\hline
Thousands & Hundreds & Tens & Ones \\
\hline
3 & 7 & 2 & 1 \\
\hline
\end{tabular}
\end{center}

\subsection{Powers of 10}
Recall that:

\[
10 \times 10=100 \text { and } 10 \times 10 \times 10=1000 \text { and } 10 \times 10 \times 10 \times 10=10000
\]

We can record that there is 1 factor of 10 in 10, 2 factors of 10 in 100, 3 factors of 10 in 1000 and so on, by writing:

\[
\begin{aligned}
10 & =10^{1} \\
100 & =10^{2} \\
1000 & =10^{3} \\
10000 & =10^{4}
\end{aligned}
\]

\begin{examples}
    \begin{questions}
        \Question[3] Find the correct number for each box.
        \begin{parts}
            \part \(120=12 \times 10^\square\)
            \part \(700=7 \times 10^\square\)
            \part \(340000=34 \times 10^{\square}\)
        \end{parts}
    \end{questions}
\end{examples}

Notation using powers of 10 is very useful in describing the place values of digits. It is particularly useful for large numbers. When we use powers of 10 to show the place value of the digits in a number, we say that the number is written in expanded form.

For example, 30721 is written in expanded form as:

\[
3 \times 10^{4}+7 \times 10^{2}+2 \times 10^{1}+1
\]

\begin{examples}
    \begin{questions}
        \Question Write each of the following numbers in expanded form and give the place value of the digit 5 .\\
        \begin{parts}
            \part \(523\)
            \begin{solutionordottedlines}[1in]
                $500+20+3=5 \times 10^{2}+2 \times 10+3$
                The place value of the digit 5 is 500 .
            \end{solutionordottedlines}
            \part \(750987\)
            \begin{solutionordottedlines}[1in]
                $7 \times 10^{5}+5 \times 10^{4}+9 \times 10^{2}+8 \times 10^{1}+7]$
                The place value of the digit 5 is 50000 .
            \end{solutionordottedlines}
        \end{parts}
        \Question Write down all the three-digit numbers that can be formed from the digits 3,7 and 9 (use each digit only once in each number formed), and list the numbers from largest to smallest.
        \begin{solutionordottedlines}[1in]
            There are 6 such numbers. They are 973, 937, 793, 739, 397, 379.
        \end{solutionordottedlines}
    \end{questions}
\end{examples}

\subsection{Large Numbers}
There are common names for some large numbers:

\[
\begin{aligned}
& 1000000=10^{6} \text { is } 1 \text { million } \\
& 1000000000=10^{9} \text { is } 1 \text { billion } \\
& 1000000000000=10^{12} \text { is } 1 \text { trillion }
\end{aligned}
\]

Large numbers are often used in astronomy. Here are some examples:

\begin{itemize}
  \item The average distance from the Earth to the Sun is approximately 150 million kilometres.
  \item There are between 100 billion and 2000 billion stars in the Milky Way.
  \item The star Sirius is approximately 75684 billion kilometres from Earth.
\end{itemize}

\subsection{Place value}
\begin{itemize}
  \item Each digit in a number has a place value.
\end{itemize}

For example, in the number 567 , the place value of 5 is 500 , the place value of 6 is 60 and the place value of 7 is 7 .

\begin{itemize}
  \item A number can be written in expanded form to show all the place values. For example: \(567=5 \times 10^{2}+6 \times 10^{1}+7\)
\end{itemize}

\begin{exercises}
    \begin{questions}
        \Question[8] Find the correct number for each box.
        \begin{multicols}{2}
        \begin{parts}
            \part \(90=9 \times 10^\square\)
            \part \(500=5 \times 10^\square\)
            \part \(1800=18 \times 10^\square\)
            \part \(20000=2 \times 10^\square\)
            \part \(45000=45 \times 10^{\square}\)
            \part \(234000=234 \times 10^{\square}\)
            \part \(7000000=7 \times 10^{\square}\)
            \part \(25000000=25 \times 10^{\square}\)
        \end{parts}
        \end{multicols}
        \Question[6] Write each of the following numbers in expanded form and give the place value of the digit 6 .
        \begin{parts}\begin{multicols}{2}
            \part 46
            \begin{solutionordottedlines}[1in]
            \end{solutionordottedlines}
            \part 623
            \begin{solutionordottedlines}[1in]
            \end{solutionordottedlines}
            \part 569
            \begin{solutionordottedlines}[1in]
            \end{solutionordottedlines}
            \part 63
            \begin{solutionordottedlines}[1in]
            \end{solutionordottedlines}
        \end{multicols}\end{parts}
        \Question[6] Write each of the following numbers in expanded form and give the place value of the digit 3 .
        \begin{parts}
            \part 2083
            \begin{solutionordottedlines}[1in]
            \end{solutionordottedlines}
            \part 3758
            \begin{solutionordottedlines}[1in]
            \end{solutionordottedlines}
            \part 5036
            \begin{solutionordottedlines}[1in]
            \end{solutionordottedlines}
            \part 43170
            \begin{solutionordottedlines}[1in]
            \end{solutionordottedlines}
        \end{parts}
        \Question[2] Write down all of the three-digit numbers that can be formed from the digits 2, 5 and 9 (use each digit only once in each number formed), and list them from largest to smallest.
            \begin{solutionordottedlines}[1in]
            \end{solutionordottedlines}
        \Question[2] Write down all of the three-digit numbers that can be formed from the digits 2,5 and 9 (each digit can be used more than once in each number formed), and list them from largest to smallest.
            \begin{solutionordottedlines}[1in]
            \end{solutionordottedlines}
    \end{questions}
\end{exercises}
