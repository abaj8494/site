In case there was any ambiguity these are the whole numbers:
\[0,1,2,3,4,...,\infty\]
\begin{itemize}
    \item $0$ is the first \emph{whole number}
    \item there is no \emph{last} whole number
    \item whole numbers are also known as \textsl{counting numbers} or \textsl{natural numbers}
\end{itemize}

Now that there is no more confusion about what a whole number \textsc{is}, we can focus on multiplying and dividing them with each other.

\begin{doublespace}
You will quickly realise that \emph{multiplying} 2 whole numbers \fillin[9] and \fillin[10] will always produce another \fillin[whole] number. However, \emph{\mbox{dividing}} two such whole numbers \fillin[does not always guarantee a whole number as a result][4in].
\end{doublespace}


Let us now begin with our study of \textbf{Multiplication:}
