\begin{doublespace}
    Division can be a tricky topic - much like \raisebox{-.6ex}{\em negative numbers}. However, a careful study at the start will pay you dividends in the future.
\end{doublespace}

An interesting place to start might be $\displaystyle \frac{4}{0}$. What does it mean to divide a number by zero?
\begin{solutionordottedlines}[1in]
    You can't do this. It doesn't make any sense. The operation is not well-defined. The calculator breaks, computer programs crash unexpectedly.
\end{solutionordottedlines}

Well let's try dividing by another whole number then: $\frac{4}{2} = \fillin[2]$. This method of writing division as a fraction is common practice beyond \emph{primary school}. We shall try to express division in this form as often as we can.

You can and should think about division as taking a group (the numerator) and dividing it into parts (the denominator). If we have 15 people and want to split them into 3 groups, each group has \fillin[5] people.

\subsection{Division without Remainders}
This clean division, i.e. that without remainders is exactly the reverse of multiplication:

\begin{examples}
\begin{questions}
    \question Fill in each box to give the equivalent multiplication or division statement.
    \begin{parts}
        \Part[1] \(60 \div 5=12\) is equivalent to \(60=12 \times \square\)
        \Part[1] \(24 \div \square=4\) is equivalent to \(24=6 \times 4\)
    \end{parts}
    \Question[1] How many equal groups of 5 objects can 15 objects be divided into?
    \begin{solutionorbox}[1in]
    \end{solutionorbox}
    \Question[1] There are 60 chocolates to be packed into boxes so that each box has 12 chocolates in it. How many boxes are needed?
    \begin{solutionorbox}[1in]
    \end{solutionorbox}
    \Question[1] A box of 72 chocolates is to be divided equally between 9 people. How many chocolates does each person get?
    \begin{solutionorbox}[1in]
    \end{solutionorbox}
    \Question[1] \(\displaystyle\frac{25}{5}=\fillin[5]\)
    \Question[1] \(\displaystyle\frac{240}{3}=\fillin[80]\)
\end{questions}
\end{examples}

\subsection{Division \emph{with} remainders}
You should still consider this as taking a group of things and dividing it by a denominator amount of parts:


This shows that \(28=3 \times 9+1\). We say ' \(28 \div 3\) equals 9 with remainder 1 '.

In this process, 28 is called the dividend, 3 is the divisor, 9 is the quotient and 1 is the remainder. The remainder must be less than the divisor.\marginpar{\raggedright The naming of things is termed \emph{nomenclature}}

\begin{examples}
    \begin{questions}
        \Question[2] Put the quotient in the first box and the remainder in the second box to make each statement true.\\
        \begin{parts}
            \part \(26=\square \times 4+\square\)
            \part \(34=\square \times 3+\square\)
        \end{parts}
    \end{questions}
\end{examples}

\subsection{The Distributive Law}
This part is interesting---it is the same as before, but very rarely are people able to do this mentally for division:

\[
\begin{aligned}
16 \div 2 & =(10+6) \div 2 \\
& =10 \div 2+6 \div 2 \\
& =5+3 \\
& =8
\end{aligned}
\]

Here is another example, this time involving subtraction. It uses the distributive law of division over subtraction.

\[
\begin{aligned}
196 \div 4 & =(200-4) \div 4 \\
& =200 \div 4-4 \div 4 \\
& =50-1 \\
& =49
\end{aligned}
\]

\begin{examples}
    The distributive law for division over addition and subtraction makes it easier to carry out some divisions.
    \begin{questions}
        \Question[2] Use the distributive law to evaluate each of the following. I'll pay double marks for these.
        \begin{parts}
            \part \((100+55) \div 5=\fillin[]\)
            \part \((200-15) \div 5=\fillin[]\)
            \part \(540 \div 5=\fillin[]\)
        \end{parts}
    \end{questions}
\end{examples}

\begin{exercises}
    \begin{questions}
        \Question[1] Fill in each box to give the equivalent multiplication or division statement.
        \begin{parts}
            \part \(108 \div 9=12\) is equivalent to \(108=12 \times \square\).
            \part \(200 \div 10=20\) is equivalent to \(\square=10 \times 20\).
            \part \(72 \div \square=12\) is equivalent to \(72=12 \times \square\).
        \end{parts}
        \Question[6] Work from left to right to calculate the following.
        \begin{parts}
            \part \(24 \times 3 \div 3\)\\
            \begin{solutionordottedlines}[1in]
            \end{solutionordottedlines}
            \part \(10 \times 2 \div 2\)\\
            \begin{solutionordottedlines}[1in]
            \end{solutionordottedlines}
            \part \(36 \div 4 \times 4\)\\
            \begin{solutionordottedlines}[1in]
            \end{solutionordottedlines}
            \part \(56 \div 8 \times 8\)\\
            \begin{solutionordottedlines}[1in]
            \end{solutionordottedlines}
            \part \(18 \div 3 \times 3\)\\
            \begin{solutionordottedlines}[1in]
            \end{solutionordottedlines}
            \part \(24 \div 12 \times 12\)
            \begin{solutionordottedlines}[1in]
            \end{solutionordottedlines}
        \end{parts}
        \Question[2] There are 28 chocolates to be divided equally among 4 people. How many chocolates does each person get?
            \begin{solutionordottedlines}[1in]
            \end{solutionordottedlines}
        \Question[2] There are 84 people at a club meeting. The organiser wishes to form 7 equal groups. How many people will there be in each group?
            \begin{solutionordottedlines}[1in]
            \end{solutionordottedlines}
        \Question[6] Fill in the boxes to make each statement true, with the smallest possible remainder.
        \begin{parts}
            \part \(17=\square \times 3+\square\)
            \part \(37=\square \times 5+\square\)
            \part \(13=\square \times 2+\square\)
            \part \(87=\square \times 8+\square\)
            \part \(41=\square \times 5+\square\)
            \part \(148=\square \times 12+\square\)
        \end{parts}
        \Question[2] Draw a dot diagram to show \(30 \div 8=3\) with remainder 6 or, equivalently, \(30=8 \times 3+6\).
            \begin{solutionorbox}[1in]
            \end{solutionorbox}
        \Question[2] Draw a dot diagram to show \(20 \div 6=3\) with remainder 2 or, equivalently, \(20=6 \times 3+2\).
            \begin{solutionorbox}[1in]
            \end{solutionorbox}
        \Question[4] Illustrate each expression on a number line.
        \begin{parts}
            \part \(7 \div 2\)\\
            \begin{solutionorbox}[1in]
            \end{solutionorbox}
            \part \(13 \div 3\)
            \begin{solutionorbox}[1in]
            \end{solutionorbox}
        \end{parts}
        \Question[6] Evaluate:
        \begin{parts}
            \part \[\frac{20}{10}\]
            \begin{solutionordottedlines}[1in]
            \end{solutionordottedlines}
            \part \[\frac{30}{6}\]
            \begin{solutionordottedlines}[1in]
            \end{solutionordottedlines}
            \part \[\frac{42}{7}\]
            \begin{solutionordottedlines}[1in]
            \end{solutionordottedlines}
            \part \[\frac{144}{12}\]
            \begin{solutionordottedlines}[1in]
            \end{solutionordottedlines}
            \part \[\frac{36}{4}\]
            \begin{solutionordottedlines}[1in]
            \end{solutionordottedlines}
            \part \[\frac{120}{3}\]
            \begin{solutionordottedlines}[1in]
            \end{solutionordottedlines}
        \end{parts}
        \Question[3] Perform each calculation by using the method indicated.
        \begin{parts}
            \part \(448 \div 32\) (divide by 2 five times)
            \begin{solutionordottedlines}[1in]
            \end{solutionordottedlines}
            \part \(640 \div 80\) (divide by 10 and then by 8 )
            \begin{solutionordottedlines}[1in]
            \end{solutionordottedlines}
            \part \(805 \div 35\) (divide by 7 and then by 5 )
            \begin{solutionordottedlines}[1in]
            \end{solutionordottedlines}
        \end{parts}
        \Question[6] Evaluate each expression by using the distributive law.
        \begin{parts}
            \part \((600+35) \div 5\)
            \begin{solutionordottedlines}[1in]
            \end{solutionordottedlines}
            \part \((300-25) \div 5\)
            \begin{solutionordottedlines}[1in]
            \end{solutionordottedlines}
            \part \(390 \div 5\)
            \begin{solutionordottedlines}[1in]
            \end{solutionordottedlines}
            \part \((600+27) \div 3\)
            \begin{solutionordottedlines}[1in]
            \end{solutionordottedlines}
            \part \((300-24) \div 3\)
            \begin{solutionordottedlines}[1in]
            \end{solutionordottedlines}
            \part \(390 \div 3\)
            \begin{solutionordottedlines}[1in]
            \end{solutionordottedlines}
        \end{parts}
    \end{questions}
\end{exercises}
