\begin{student}
This is the stage upon which much of the mathematics you will see in High-School is set:

\begin{center}
\begin{tikzpicture}[scale=0.8]
    \centering
    \draw[<-] (-7.5,0) -- (7.5,0) node [left] at (-7.5,0) {$-\infty$};
    \draw[->] (-7.5,0) -- (7.5,0) node [right] {$+\infty$};
    \foreach \x in  {-6,-4,-2,0,2,4,6}
        \draw[shift={(\x,0)},color=black] (0pt,3pt) -- (0pt,-3pt);
    \foreach \x in {-3,-2,-1,0,1,2,3}
        \draw[shift={(2*\x,0)},color=black] (0pt,0pt) -- (0pt,-3pt) node[below] {$\x$};
\end{tikzpicture}
\end{center}

Then if we overlay another axis perpendicular to this one we get:

\begin{center}
\begin{tikzpicture}[scale=0.4]
    % Horizontal axis
    \draw[->] (-7.5,0) -- (7.5,0);
    \draw[<-] (-7.5,0) -- (7.5,0);
    \node[left] at (-7.5,0) {$-\infty$};
    \node[right] at (7.5,0) {$+\infty$};

    \foreach \x in {-6,-4,-2,0,2,4,6}
    \draw[shift={(\x,0)},color=black] (0pt,3pt) -- (0pt,-3pt);

    \foreach \x in {-3,-2,-1,1,2,3}
    \draw[shift={(2*\x,0)},color=black] (0pt,0pt) -- (0pt,-3pt) node[below] {$\x$};

    % Vertical axis
    \draw[->] (0,-7.5) -- (0,7.5);
    \draw[<-] (0,-7.5) -- (0,7.5);
    \node[below] at (0,-7.5) {$-\infty$};
    \node[above] at (0,7.5) {$+\infty$};

    \foreach \y in {-6,-4,-2,0,2,4,6}
    \draw[shift={(0,\y)},color=black] (3pt,0pt) -- (-3pt,0pt);
    \node[below right] at (0,0) {$0$};

    \foreach \y in {-3,-2,-1,1,2,3}
    \draw[shift={(0,2*\y)},color=black] (0pt,0pt) -- (-3pt,0pt) node[left] {$\y$};

\end{tikzpicture}
\end{center}

And finally, placing yet \emph{another} axis perpendicular to both of these we spawn 3-Dimensional Space!

\begin{center}
\begin{tikzpicture}[scale=0.8]
\centering

% Set up 3D view
\tdplotsetmaincoords{60}{110}

% Axes
\draw[->] (-4,0,0) -- (4,0,0);
\draw[<-] (-4,0,0) -- (4,0,0);
\draw[->] (0,-4,0) -- (0,4,0);
\draw[<-] (0,-4,0) -- (0,4,0);
\draw[->] (0,0,-6) -- (0,0,6);
\draw[<-] (0,0,-6) -- (0,0,6);

% Horizontal axis (x)
\foreach \x in {-3,-2,-1,1,2,3}
\draw[shift={(\x,0,0)},color=black] (0pt,2pt) -- (0pt,-2pt) node[below] {$\x$};
\node[right] at (4,0,0) {$+\infty$};
\node[left] at (-4,0,0) {$-\infty$};

% Vertical axis (y)
\foreach \y in {-3,-2,-1,1,2,3}
\draw[shift={(0,\y,0)},color=black] (2pt,0pt) -- (-2pt,0pt) node[left] {$\y$};
\node[above] at (0,4,0) {$+\infty$};
\node[below] at (0,-4,0) {$-\infty$};

% Perpendicular axis (z)
\foreach \z in {-5,-4,-3,-2,-1,1,2,3,4,5}
\draw[shift={(0,0,\z)},color=black] (2pt,0pt) -- (-2pt,0pt);
\node[above right] at (0,0,8) {$+\infty$};
\node[below left] at (0,0,-8) {$-\infty$};
\end{tikzpicture}
\end{center}
\end{student}

\begin{questions}
    \Question[1] The highest possible dimensional space is \fillin[$\infty$]!
\end{questions}

\begin{student}
\subsection{Less than \& Greater than}
On the number line, you can compare 2 quantities:
\begin{center}
    \begin{tikzpicture}[scale=0.8]
        \draw[->] (-6, 0) -- (6,0);
        \draw[<-] (-6, 0) -- (6,0);
        \foreach \x in {-4,-2,0,2,4}
            \draw[shift={(\x,0)},color=black] (0pt,3pt) -- (0pt,-3pt);
        \node[above] at (4,0) {$a$};
        \node[above] at (-2,0) {$b$};
        \node[below] at (0,0) {$0$};
        \draw (1,0) circle;
        \draw (-2,0) circle;
    \end{tikzpicture}
\end{center}
\vspace*{-0.3cm}
\begin{align*}
    a &> b\\
    b &< a
\end{align*}
\end{student}

\begin{examples}
    \begin{questions}
        \Question[1] Draw a number line and on it mark with dots all the whole numbers less than 5.
        \begin{solutionorbox}[2in]
            \begin{center}
            \end{center}
        \end{solutionorbox}
        \Question[1] Draw another number line and on it mark all the whole numbers greater than 45 and less than 52.
        \begin{solutionorbox}[2in]
            \begin{center}
            \end{center}
        \end{solutionorbox}
    \end{questions}
\end{examples}


\begin{exercises}
    \begin{questions}
        \Question[4] For each of the following, draw a number line from 0 to 10.
        \begin{parts}
            \part Mark the numbers 2,4,6 and 8 on it.
            \begin{solutionorbox}[1in]
            \end{solutionorbox}
            \part Mark the numbers 1,3,5 and 7 on it.
            \begin{solutionorbox}[1in]
            \end{solutionorbox}
            \part Mark the whole numbers less than 5 on it.
            \begin{solutionorbox}[1in]
            \end{solutionorbox}
            \part Mark the whole numbers less than 8 and greater than 2 on it.
            \begin{solutionorbox}[1in]
            \end{solutionorbox}
        \end{parts}
        \Question[2] The manager of an underground railway system decides to save time in the mornings by having one particular train only stop at every third station between stations 1 and 19. The stations are all \(1 \mathrm{~km}\) apart. Show the stations on a number line and mark with a dot each station where the train stops.
            \begin{solutionorbox}[2in]
            \end{solutionorbox}
    \end{questions}
\end{exercises}
